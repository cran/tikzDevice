% !TEX TS-program = Sweave
% !TEX root = tikzDevice.Rnw

% Font macros
\newcommand{\lang}{\textsf}
% Mbox it to prevent hyphenation of code
\newcommand{\code}[1]{\mbox{\ttfamily #1}}
\newcommand{\pkg}{\textbf}

% Other LaTeX macros
\renewcommand{\sectionautorefname}{Section}
\renewcommand{\subsectionautorefname}{Subsection}

% Text macros
\newcommand{\TikZ}{Ti\textit{k}Z}
\newcommand{\tikzDeviceVersion}{0.5.0-{\bfseries\color{red}Beta}}

% Source code formatting macros
\lstloadlanguages{R}

\lstset{basicstyle = \ttfamily}

\lstdefinestyle{sweavechunk}{
	language = R,
	% The hacked version of the Sweave output function inserts these tags
	% around each output portion- we can grab them and apply formatting
	% to everything inside.
	moredelim=[is][\color{gray}]
		{swe@veSt@rtOutput}
		{swe@veEndOutput},
	% Normal TeX tildes are kind of ugly- let's substitute a math symbol instead.
	% Shit, why not replace <- as well? *Mad typographic scientist cackle*
	literate={~}{{$\sim$}}1,
	showstringspaces = false,
	upquote = false,
	commentstyle = {\color{blue!70}\itshape}
}

\lstdefinestyle{latexsource}{
	language = [LaTeX]TeX,
	showstringspaces = false,
	upquote = false,
	commentstyle = {\color{red!80}\itshape}
}

\lstdefinestyle{latexexample}{
	language = [LaTeX]TeX,
	showstringspaces = false,
	upquote = false,
	commentstyle = {\color{red!80}\itshape},
	moredelim=[is][\color{red}]{XX}{XX}
}

\lstdefinestyle{bashsource}{
	language = bash,
	% literate={~}{{$\sim$}}1,
	moredelim=[is][\color{gray}]
		{!out}
		{!/out},
	showstringspaces = false,
	upquote = false,
	commentstyle = {\color{blue!80}\itshape}
}


% TikZ Style definitions.
\tikzset{
	% Taken from one of the first examples in the PGF manuel.
	package warning/.style={
		rectangle split,
		rectangle split parts = 2,
		rounded corners,
		draw = red!50,
		thick,
		fill = red!10, 
		inner sep = 1ex,
		text width = \textwidth
	}
}


% TikZ Macros.

\newcommand{\tikzDocDisclaim}[2]{

	\begin{tikzpicture}
		
		\node[package warning]{
			\begin{center}
				\large\bfseries
				#1	
			\end{center}	
			\nodepart{second}		
				#2	
		};

	\end{tikzpicture}

}
